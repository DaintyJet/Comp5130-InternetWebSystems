\documentclass{article}


\usepackage{graphicx} % Required for inserting images
\usepackage{a4wide}   % Width
\usepackage{hyperref} % Hyper links
\usepackage{tabularx} % More Tables 
\usepackage{subcaption}


\graphicspath{{Images/}}

% Config Links
\hypersetup{
    colorlinks=true,
    linkcolor=blue,
    filecolor=magenta,      
    urlcolor=cyan,
    pdftitle={Overleaf Example},
    pdfpagemode=FullScreen,
}


% % Redefine \section to be centered and unnumbered
% \titleformat{\section}
%   {\normalfont\Large\bfseries\centering} % Format: Large, bold, centered
%   {\thesection } % No section numbering
%   {0pt} % No extra space before title
%   {} % No extra code before the title

% % Redefine \subsection to be centered and unnumbered
% \titleformat{\subsection}
%   {\normalfont\large\bfseries\centering} % Format: large, bold, centered
%   {} % No subsection numbering
%   {0pt} % No extra space before title
%   {} % No extra code before the title

\title{Internet \& Web Systems Week 5 Progess}
\author{by Matthew Harper }
\date{October 2024}

\begin{document}

\maketitle
\section{Progress}
Rather than work on the stretch goals from the previous cycle, which for the most part can be done at any time I decided to implement some of the
core feature I need to start implementing the vulnerabilities in the site. Namely I needed to start implementing authentication, decide on a database
scheme, expand the stubs for the APIs, and start to plan out how much logging will occur at each level and where it may be useful.

I was able to implement authentication, although I originally planned to implement this using passport, for now I was able to implement a system that stores
usernames and hashed passwords that were hashed with the \textit{bcrypt} function. The later authentication of users is handled with JSON Web Tokens, as of
now it uses a placeholder secret, but I will change that to be controlled in the \textit{.env} file so it can be an added vulnerability.

I also started to create the Database wrapper, as part of this I explored using \textit{mongoose} which seems to be a popular library to make \textit{MongoDB}
operations easier. Otherwise I only implemented the user authentication database through this, I have planned out the other two databases one being a UID
mapper so I can more easily induce or reduce vulnerabilities and the final one being a simple database used to store the user supplied data.

Finally I added the JSON Web Token check to most of the API endpoints and user pages, I also expanded the API stubs with the basic information I will require
as part of the request bodies in order to complete the given task. At this time it simply prints the information out. I did not and do not plan to do much
with regards to the frontend until I have to, as that is the least important in terms of functionality for me and this project.

I also started to formalize the vulnerabilities that I can implement. However, that is a rolling task that will occur week-to-week.

\section{Not Completed}
The primary goal was to implement authentication, design the collections used in the database and start implementing the APIs. One of my stretch goals
was to implement all of the database collections, however the only one that occurred was the authentication database as that was required. I did not
implement all of the APIs, however I was able to add the authentication checks to all the user endpoints and apis. As the Admin pages will need to
check if the user is an admin I will implement the authentication scheme there later, currently I believe the easiest method is the admin accounts will
have a \textit{UID < 1000} and normal users will have a UID that is larger. This is basic, and allows me to implement later vulnerabilities. So I was
able to add authentication for normal users and some of the APIs, but I was unable to finish the admin authentication.

I also wanted to explore enabling HTTPs, however that was delayed due to other classwork requiring a significant amount of time.
\end{document}
